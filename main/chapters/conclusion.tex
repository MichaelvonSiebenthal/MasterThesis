
  This chapter includes the conclusion and provides an outlook for future
  research.

  \section{Conclusion}
  \label{section:conclusion}

  The aim of this thesis is to assess to what extent semi-synthetic graphs based 
  on real feature data are useful for machine learning. This aim corresponds to
  the research question stated in section \ref{section:research_topics}. The
  results for answering this question are mixed and are based on the data
  presented in chapter \ref{section:data}, the results of chapter
  \ref{section:results} and the discussion in chapter \ref{section:discussion}.
  \\

  \noindent Graph machine learning on the graph created from the Bank 
  Telemarketing dataset fails to overcome the problem of unbalanced labels. It 
  however yields similar results in terms of accuracy as the standard machine 
  learning models. In addition it is shown, that if a network structure can be 
  created which corresponds to the label, that the problem of unbalanced label 
  data could be overcome. \\

  \noindent GraphSage performs well for the US Airline Passenger dataset and
  is second only to the random forest classifier in terms of accuracy. As
  discussed, following the principle of Occam's razor, the good results for the
  GraphSage model must be taken with a grain of salt. For the purpose of
  gaining customer insights, simpler models such as ANN, SVM or AdaBoost provide
  only marginally inferior results and are more practical and thus preferable
  for this setting. Given similar results for more sensitive applications such
  as medicine, the GraphSage model could be preferred as even marginal
  performance improvements can be of utmost importance. In short, the GraphSage
  yields competitive results in terms of accuracy, the results are however not
  superior enough to necessarily warrant the complex graph generation process
  and model complexity of GraphSage. Lastly, the graph convolutional network and
  Node2Vec are not successful for the given classification task. For that
  reason more recent graph machine learning models should be tested and/or new 
  models should be developed to expand the possibilities of graph machine 
  learning.\\

  \noindent The perhaps most interesting result is provided by the US Airline
  Passenger graph plots shown in figure \ref{fig:us_airline_nodes}. The MAG 
  model successfully generates neighborhoods within the graph, where the nodes 
  are grouped according to their similarity which respects the similarities 
  between all attributes. This observation is further confirmed by the scatter 
  plots shown in figure \ref{fig:node2vec}. The graphs and the scatter plots
  make it possible to visually interpret the relationships within the data. 
  Standard scatter plots using the feature data do not allow for the generation 
  of such insightful graphs/plots. \\

  \noindent To summarize and provide an answer to the research question, it is
  shown that yes, semi-synthetic graphs can be useful for machine learning in a
  classification setting. The graph machine learning models are shown to be 
  competitive and could potentially even provide a solution for overcoming the 
  difficulties associated with unbalanced label data. This however requires the 
  availability of a graph with a structure which corresponds to the label. The 
  usefulness of semi-synthetic graphs are limited by the complexity associated 
  with generating graphs and the general complexity of graph machine learning 
  models. The graph based models further fail to outperform the standard machine 
  learning methods and are ranked anywhere between the second to fifth best 
  method depending on how one weights the trade off between model complexity 
  and accuracy. Based on this, the hypotheses presented in section
  \ref{section:research_topics} can be answered as follows: \\

  \noindent\textbf{H1:} Graph machine learning using semi-synthetic graphs fail
  to outperform standard machine learning methods. This hypothesis is thus
  rejected. \\

  \noindent\textbf{H2:} Graph machine learning is shown to be a competitive
  strategy with competitive results in line with the results shown for the
  standard machine learning methods. This hypothesis is not rejected.

  \section{Outlook}

  The discussion in chapter \ref{section:discussion} and the conclusion in section
  \ref{section:conclusion} provides interesting topics for future research.
  These topics include graph generation, cluster analysis on graphs and graph 
  machine learning models. 

  \paragraph{Graph Generation} \mbox{}
  
  \noindent Creating semi-synthetic graphs using the MAG method is shown to be a 
  viable method. The graph is however sensitive to attribute selection and 
  setting appropriate link-affinity probabilities. A fist step for resolving
  this issue was introduced by \cite{kim2011modeling} as a follow-up to their
  MAG model. In this paper they propose a reverse model, in which given a real
  graph, the attributes and the link-affinity probabilities are estimated such 
  that the attributes and link-affinity probabilities generate the observed real
  graph. This is however only a partial solution to the task given for this
  master's thesis. The estimated attributes do not correspond to real 
  attributes/features. In this model, the attributes are estimated to fit the 
  graph. An interesting topic for future research would be to develop a model 
  that generates a semi-synthetic graph based on attributes which at the same
  time optimizes link-affinity probabilities such that the resulting graph
  adheres to network properties observed in real graphs. Perhaps these more
  realistic graph properties can improve the performance for graph machine
  learning. This is an area worth consideration as especially the degree
  distribution and centrality measures shown in figure 
  \ref{fig:centrality_measures} force graph machine learning models to consider
  a large number of neighbors. In addition, even when sampling, the
  neighborhood is bound to always have the same size, given the range of the
  degree distribution shown in figure \ref{fig:centrality_measures}. This is a
  potential limiting factor for graph machine learning, as it potentially makes
  it more difficult to distinguish- and make use of different structures
  withing the graph.

  \paragraph{Cluster Analysis on Graphs} \mbox{}

  \noindent The graph plots shown in figure \ref{fig:us_airline_nodes} reveal,
  that semi-synthetic graphs can be excellent tools for visualizing data. The
  amount of useful information provided for interpreting the data is a welcome
  and unexpected result. For understanding the relationships between the
  attribute data and the label, the graph plots yielded the most useful
  information for identifying relationships within the data. An interesting
  topic for future research could be to assess how common clustering methods
  such as k-means, fuzzy clustering or CLIQUE could be used for analyzing
  graphs. Perhaps better and more graph centric methods could be developed. An
  excellent overview of existing graph clustering methods is provided by
  \cite{zhou2020graph}. Their article can be used as an initial reference point
  for identifying new applications of existing graph clustering methods or for 
  developing new models.

  \paragraph{Graph Machine Learning Models}\mbox{}

  \noindent Graph convolutional networks and GraphSage are selected for this
  thesis as they are probably the two most well-known GNNs. There are however 
  newer and more sophisticated models such as \ac{gat} \citep{velivckovic2018graph} 
  or \ac{gin} \citep{xu2019powerful}. \\

  \noindent GAT models have the ability of identifying more- or 
  less important neighbors in the graph. This is a useful ability and has been 
  shown to improve performance in some cases. This ability is unfortunately 
  most likely limited by the very large number of neighbors present in the US Airline 
  Passenger graph as shown in figure \ref{fig:centrality_measures}. For that
  reason, more realistic graphs with smaller number of degrees should be
  generated as suggested in the previous graph generation paragraph. \\

  \noindent GIN models are very well suited for distinguishing structures 
  within the graph. The authors \cite{xu2019powerful} present the general GIN
  framework which can accept any type of features as inputs whilst ensuring
  that the aggregation function is injective. Given the injective aggregation
  strategy, it is shown that the GIN can be as expressive at distinguishing
  graph structures as the Weisfeiler-Lehman graph isomorphism test 
  \citep{weisfeiler1968}. Again, for such a model to work best, the number of
  degrees would most likely need to be smaller than currently present in the
  graph of the US Airline Passenger dataset. Given the large number of degrees,
  the neighborhoods of every node are currently of the same size when using the
  sampling strategy outlined for the GraphSage model. \\

  \noindent Given a more realistic graph with a smaller number of degrees, it
  would be interesting to develop and assess, whether a new method which
  includes the attention mechanism of the GAT network with the isomorphic 
  capabilities of the GIN method could be of use. This new model could for
  instance be tested on well understood benchmark graphs such as Cora
  \citep{mccallum2000automating} or Citeseer \citep{giles1998citeseer}. 

