	
	\section{General Overview}

	Everyone has surely made the at times inconvenient experience of having been contacted by marketers, worst of all telemarketers. Not only are these
	marketers more often than not unpleasantly persistent, they in addition often try to sell something which we are not interested in. While we may 
	never get rid off marketers, there are methods which could ensure that we at least receive offers which we are likely to be interested in. This 
	would make a positive outcome for both the customer and the marketer much more likely. \\ 

	\noindent Thanks to the advent of big data, technology companies such as Google and Facebook have long since built their entire business model on 
	providing customer data to marketers. The customer data is often collected through social media, internet searches and by any other ways and means 
	by which customer data can be collected. Of course, many marketers also possess their own databases from their customers which they may use for 
	marketing purposes. \\

	\noindent In order for marketers or from now on more generally companies to accurately predict, whether a particular client would be interested in a	particular product, different methods for predicting client interests can be applied. For instance, a naive but often time predictive method would
	be to offer a product to people which have purchased the same or a similar product in the past. While one could apply this method, it has a major 
	shortfall in that it does not allow us to identify new customers for the product. In order to receive better results, more sophisticated methods 
	from machine learning could be used. In machine learning terminology, predicting whether a client would be interested in a product is referred to 
	as a classification task. There are many machine learning methods which can perform such a task to varying success such as:

	\begin{enumerate}
		\item Linear- or Quadratic Discriminant Analysis
		\item Support Vector Machines
		\item Ensemble Methods
		\item Neural Networks
	\end{enumerate}

	\noindent Of course this list is not exhaustive and there are a myriad of methods available. Further, there is no one size fits all solution, 
	as every machine learning method has its advantages and disadvantages. The methods listed are all powerful methods in their own right and are
	capable of performing a large range of classification tasks. Traditionally, these methods were designed to perform tasks using "classical" 
	datasets such as cross sectional data. The term "classical" is a reference where an observation in a dataset stands for itself. For instance survey
	data collected from the respondent "Peter" are not directly linked/connected to the answers of the respondent "Paul". Graphs or networks are very
	different in this respect. As an illustration, observations in social networks are very much connected (e.g. Peter and Paul could be friends). In
	the past 10 years there has been an explosion of new research in this area, where data scientist have developed methods to perform machine learning 
	tasks on network data. This is an exciting new frontier in data science, which from a methodological perspective will be the focus of this master 
	thesis for performing client classification tasks. \\

	\noindent In the following sections a detailed description of the classification task for this master thesis is given, followed by a literature
	review of the extant literature. 


	\section{Topical Setting}

	Retail banking is an area in which client advisers typically serve several thousand clients. In addition, advisers typically work in teams which 
	makes personalized advice virtually impossible. For this reason, retail clients are often undeserved and are often dissatisfied with their bank. 
	This is an area where classifying clients according to their needs or interests could tremendously improve the service quality provided. 



	Another interesting application and the focus for this thesis is to classify bank clients according to their investment preference 
	(e.g. which type of products should be advertised to which client?). This would be especially useful in the retail banking segment where advisers 
	typically cannot know their clients personally due to the large number of clients being serviced. Investor classification is the intended main focus
	of this Master Thesis. \\

	\section{Challenges}

	\noindent This topic faces many different hurdles due to the low availability and mostly absence of available bank client data. To the extent 
	possible, appropriate data sets will be retrieved (thus far 1 dataset found). The main difficulty however is to find a dataset which both includes 
	attribute/feature data and the network structure of the customer data. For this reason, mostly methods to create synthetic data will be used to 
	create the dataset for the subsequent simulation/testing.


