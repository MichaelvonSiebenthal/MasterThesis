	
	The aim of this thesis is to explore the relatively new field of machine
	learning on graphs for the purpose of gaining customer insights. Graph
	machine learning is the current frontier in machine learning and has vast
	applications in many areas as recently shown by the success of AlphaFold
	\citep{senior2020improved}. AlphaFold made a breakthrough for predicting
	protein structures where they made use of the observation that a folded
	protein can be considered as a spatial graph \citep{AlphaFoldTeam2020}. In
	addition, there are a vast range of applications for graphs in the fields of 
	natural science, social science and many more as shown by the excellent 
	overview given by \cite{zhou2020graph}. In principle, graphs are useful 
	whenever one wants to model interactions or relationships.\\

	\section{Relevance to Economics}

	\noindent From a business \& economics perspective, graphs are particularly
	interesting if one wants to for instance model the interactions between
	institutions. An example for this is a study published by
	\cite{schweitzer2009economic} which created a graph showing the 
	interdependencies of international banks as a network, see figure 
	\ref{fig:bank_network}. This is a useful representation of interdependencies 
	and is an important basis for making systems more robust. \\

	\begin{figure}
		\centering
		\includegraphics[width=0.5\textwidth]{bank_network.jpg}
		\caption{Bank Network}
		\cite[p. 424]{schweitzer2009economic}
		\label{fig:bank_network}
	\end{figure}

	\noindent Another interesting application of graphs for business \& 
	economics are social interactions. While there are many different types of
	social interactions of interest, social interactions for marketing
	purposes have been among the most popular. Indeed, this is one of the main
	areas where social networks such as Facebook or search providers such as 
	Google make their revenue by selling advertising 
	\citep{Facebook2021,Alphabet2021}. Both Facebook and Google have the advantage, 
	that their businesses naturally capture relational or more generally network 
	data which can be represented as graphs. Most researchers or companies however 
	do not have access to such data. Companies for instance may have access to large
	amounts of customer data, however they typically would not have access to
	relational information (e.g. which client is connected with which other
	clients). The same is true for researchers, where social scientists
	often collect data via anonymous surveys. This is an issue in terms of data
	access. \\

	\noindent This leads us to the two main aspect of this master thesis which
	consists of:
	
	\begin{enumerate}
		\item Methodology
		\item Application
	\end{enumerate} 

	\noindent Given the difficult access to graph data, this thesis will
	explore whether (semi-) synthetic graphs can be generated using
	cross-sectional data gathered from surveys. The aim is not only to generate
	graphs but to then test whether the resulting graph can be used for
	meaningful machine learning tasks. In order to test this, cross-sectional 
	data of banking clients will be gathered. The aim in terms of application 
	will be to perform an investor classification task. This application area is
	chosen due to the author's experience in this field having worked as a
	client adviser at a bank for over 10 years. One could however choose almost
	any customer classification area, meaning that the application can be chosen
	arbitrarily within reason. \\

	\noindent It would of course be best, if one would have access to real
	graph data. If graph data had been available for a unique application, this
	would have been preferred for this thesis. The absence of such available 
	graph data and the difficulty of generating graph data sparked the interest
	for exploring synthetic graph generation for subsequent machine learning
	tasks. If this application proves to be successful, this could provide an 
	alternative and hopefully improved approach for analyzing cross-sectional 
	data. For that reasons, graph machine learning results will be compared to
	standard machine learning methods.\\

	\noindent In the following section, the motivation for choosing bank
	clients (retail clients) for the classification task is presented. 


	\section{Application}

	Retail banking is an area in which client advisers typically serve several 
	hundred if not thousands of clients. In addition, advisers typically work
	in teams which makes personalized advice virtually impossible. For this 
	reason, it is impossible for an adviser to ensure that she provides offers
	to her clients which are in line with their interests. From personal
	experience of the author, having worked as an adviser for over 10 years, 
	this can lead to unsatisfactory outcomes when contacting clients for services or 
	products that they are not interested in. This is largely due to inadequate
	client selection practices. In a bank setting, clients for marketing 
	campaigns, such as promoting investment products, are typically selected based
	on whether they have available liquidity for investment and/or whether a 
	client has already invested in similar products. While this selection makes
	sense to a certain extent, it falls short in that there remain often too 
	many uninterested clients in the selection. In addition, people who might 
	be interested in the product being promoted might be falsely excluded given 
	the rather simple selection criteria. \\
	

	\noindent This is an area where classifying clients according to their 
	interests could be of great value and where machine learning methods could 
	be of particular use. It would in addition improve the service quality 
	rendered to clients, prevent unnecessary and perhaps annoying marketing 
	calls and improve the efficiency of a marketing campaigns. \\

	\noindent In the following subsection a literature review regarding bank
	client classification is given. The literature review for graph generation
	and graph machine learning methods are provided in
	chapter~\ref{section:theory}. The literature review for graph generation
	requires some theoretical background in graph theory, which is why this
	part of the literature review is only given in the following chapter.  
	
	\subsection{Literature Review Application}

	There is a lot of published research regarding bank client classification
	using a myriad of methods. A recent article published by
	\cite{sukharev2020ews} made use of graph neural networks for credit scoring
	bank clients. The authors were successful in developing a model in which
	the bank clients cash flows were used to link bank clients with each other
	in order to generate a graph. They show in their paper, that by exploiting
	these relationships, they were able to develop a significantly improved
	credit scoring model. This article is interesting in that they are among of
	the first to use graph machine learning for classifying bank clients. In
	addition, \cite{sukharev2020ews} show a unique advantage of banks in terms
	of available data. Thanks to the large increase in use of online banking, as
	shown for the case of Switzerland \citep{Statistapayment}, there is an
	abundace of relational information available for generating graphs.
	Unfortunately banks are hesitant to release such data for external research 
	due to privacy concerns or bank secrecy laws. \\

	\noindent Another interesting paper by \cite{weber2018scalable} focuses on 
	anti-money laundering (AML) detection tasks which is a very important and 
	labor intensive task for banks. The authors employ graph machine learning
	techniques to predict the plausability of a transaction between two parties
	(link prediction task). \\

	\noindent Lastly the study by \cite{moro2014data} analyzed a Portugese bank client
	data set in which clients were selected for a telemarketing campaign. The
	dataset contains a set of features of the contacted clients and their
	response. After the campaign, clients either invested into a long-term
	deposit or not. The study is thus set up as a binary classification
	problem. This study and dataset is well known and has been used by many
	authors as a test dataset to compare the performance of machine learning
	methods. To the authors knowledge, the dataset has not been used for
	synthetic graph creation or graph machine learning tasks. For that reason,
	this dataset will be used as a reference for the analyses in the upcoming
	sections. \\

	\noindent In terms of studies which perform investor classification, no
	such studies have been found that use machine learning on graphs. In that
	sense this master thesis is unique. In the following section, the required
	theoretical background for this thesis is introduced. 


