	\section{General Overview}

	The aim of this Master Thesis is to analyze to what extent machine learning methods on graphs such as embedding strategies like node2vec 
	\citep{grover2016node2vec} or Graph Neural Networks (GNN) \citep{kipf2016semi,hamilton2017inductive,velivckovic2017graph,vaswani2017attention} 
	can be applied for bank client classification. Graph machine learning methods have proven to be very successful for classification or prediction 
	tasks among others. 
	The advantage of these methods is that they can consider the network structure of a dataset (e.g. social network). This is a valuable feature 
	if one for instance compares it to the capabilities of traditional machine learning methods such as Convolutional Neural Networks (CNN) which can 
	"only" work with grid structures (e.g. image classification). Graph machine learning methods are currently widely used in recommendation systems 
	for social media (e.g. Pinterest, \citet{ying2018graph}) and could be similarly beneficial for banks when classifying clients. Credit Scoring for 
	instance appears to be an application in which GNN perform very well as shown by \citet{sukharev2020ews}. \\

	\section{Specific Setting}

	\noindent Another interesting application and the focus for this thesis is to classify bank clients according to their investment preference 
	(e.g. which type of products should be advertised to which client?). This would be especially useful in the retail banking segment where advisers 
	typically cannot know their clients personally due to the large number of clients being serviced. Investor classification is the intended main focus
	of this Master Thesis. \\

	\section{Challenges}

	\noindent This topic faces many different hurdles due to the low availability and mostly absence of available bank client data. To the extent 
	possible, appropriate data sets will be retrieved (thus far 1 dataset found). The main difficulty however is to find a dataset which both includes 
	attribute/feature data and the network structure of the customer data. For this reason, mostly methods to create synthetic data will be used to 
	create the dataset for the subsequent simulation/testing.


