	

	Everyone has surely made the at times inconvenient experience of having 
	been contacted by marketers, worst of all telemarketers. Not only are these		
	marketers more often than not unpleasantly persistent, they in addition 
	often try to sell something which we are not interested in. While we may 
	never get rid off marketers, there are methods which could ensure that we 
	at least receive offers which we are likely to be interested in. This 
	would make a positive outcome for both the customer and the marketer much 
	more likely. \\ 

	\noindent Thanks to the advent of big data, technology companies such as 
	Google and Facebook have long since built their entire business model on 
	providing customer insights to marketers. The customer data is often 
	collected through social media, internet searches and by any other ways and 
	means by which customer data can be collected. Of course, many marketers 
	also possess their own databases from their customers which they may use for 
	marketing purposes. \\

	\noindent This leads us to the general topic for this master thesis which 
	consists of two main components, which are:

	\begin{enumerate}
		\item Methodology
		\item Application
	\end{enumerate}


	\section{Overview Methodology}

	This thesis has a strong focus on methodology as suggested by its title.
	Machine Learning on graphs has become a hot topic in the field of machine
	learning with a large number of papers being published every year in
	prominent journals and conferences \citep{Ivanov2020}. While graph theory
	is an old field of mathematics and can be traced back to Leonhard Euler and 
	the famous "Königsberg Bridge Problem" \citep{euler1741solutio}, it has 
	experienced a recent revival in machine learning. As it is a relatively 
	novel field in machine learning, it provides for a fruitful ground for 
	testing methods and applications. \\

	\noindent Graph structures are significantly different compared to
	traditional types of data usually used for machine learning tasks or
	regression analyses. Graph structures are special in that the data points
	in a graph have connections with each other. A practical example for this
	are social networks. In a social network, the profiles of "Peter" and
	"Paul" might be connected because Peter and Paul are friends. This aspect
	is unique to graph or network data and provides both additional information
	and additional complexity to graph data not found in other data types. An
	introduction to graph theory will be given in chapter 2. 



	\section{Overview Application}

	Retail banking is an area in which client advisers typically serve several 
	hundred if not thousands of clients. In addition, advisers typically work
	in teams which makes personalized advice virtually impossible. For this 
	reason, it is impossible for an adviser to ensure that she provides offers
	to her clients which are in line with their interests. From personal
	experience of the author, having worked as an adviser for over 10 years, 
	this can lead to unsatisfactory outcomes when contacting clients for services or 
	products that they are not interested in. This is largely due to inadequate
	client selection practices. In a bank setting, clients for marketing 
	campaigns, such as promoting investment products, are typically selected based
	on whether they have available liquidity for investment and/or whether a 
	client has already invested in similar products. While this selection makes
	sense to a certain extent, it falls short in that there remain often too 
	many clients in the selection, which are not interested in the offer of a 
	specific marketing campaign or potentially interested clients are falsely 
	excluded in the pre-selection process. \\
	

	\noindent This is an area where classifying clients according to their 
	interests could be of great value and where machine learning methods could 
	be of particular use. It would in addition improve the service quality 
	rendered to clients, prevent unnecessary and perhaps annoying marketing 
	calls and improve the efficiency of a marketing campaigns.
	

	\section{Motivation}

	The topic is motivated both by the methodology and its application. As
	shown, machine learning on graphs is a very recent and promising field in
	machine learning. From an application perspective, many banks lag behind in
	terms of using appropriate customer classification strategies. It would
	therefore be of particular use to discover useful approaches for accurate
	customer classification. Thanks to the novelty of graph machine learning
	techniques, many possible applications remain to be explored. This provides
	fruitful ground for research and this master thesis. 

	\noindent In the following an overview of the extant literature regarding 
	client classification is given with an emphasis on the use of graph data. 


	\section{Literature Review}

	There is a lot of published research regarding bank client classification
	using a myriad of methods. Classifications tasks are performed in areas
	such as credit scoring, anti-money laundering (AML) compliance or marketing
	purposes among others
	\citep{sukharev2020ews,weber2018scalable,moro2014data}. \\

	CONTINUE ANOTHER TIME



