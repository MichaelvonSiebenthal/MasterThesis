
  The analysis of the data in chapter \ref{section:data} and the presentation
  of the \acs{ml} results in chapter \ref{section:results} provided a
  myriad of interesting results. These results primarily touch following topics
  and provide the structure for the discussion:

  \begin{enumerate}
      \item Informativeness of semi-synthetic graphs
      \item Appropriateness of semi-synthetic graphs for \acs{ml}
      \item \acs{gml} vs. standard \acs{ml}
  \end{enumerate}

  \section[Informativeness of Graphs]{Informativeness of Semi-Synthetic Graphs}
  \label{section:informativeness}

  The graph generated for the \acs{us} airline passenger dataset presented in 
  section \ref{section:airline_data} provided interesting results for cluster 
  analysis. The \acs{mag} model was successfully used for generating a graph that 
  captured useful neighborhood structures as shown in figure \ref{fig:us_airline_nodes}. 
  Specifically, similar nodes were grouped together based on their similarity 
  given the selected attributes for the \acs{mag} model. Especially appealing is 
  that neighborhoods are also consistent within neighborhoods. To illustrate this, 
  one can observe that all nodes are clustered together based on their 
  similarity even within the two main clusters which formed based on the 
  attribute type of travel. This observation is consistent for several 
  additional levels. For instance, nodes form neighborhoods based on age within 
  the cluster of passengers traveling for business purposes. The only attribute 
  for which no significant clusters emerged is gender. The values in the 
  link-affinity matrix for gender were set, so that a connection between the
  same gender is formed with a probability of 0.6. For observations of the
  opposite gender, the probability of a connection is 0.4. These probabilities 
  were set based on the assumption, that observations of the same gender are 
  more similar than those of the opposite gender. It is however not assumed, 
  that the passenger experience for women would be significantly different to 
  the one for men. For that reason, the resulting gender based clusters are
  only marginally pronounced to the point that no visual interpretations can be
  made. In addition, one needs to consider the distribution of gender. The
  attribute is almost perfectly 50/50 split into male and female observations.
  Assigning more extreme link-affinity probabilities would strongly alter the
  entire graph. Most likely, the graph would be further split according to
  gender. In comparison, attributes with more categories can have more extreme
  values assigned to them. Larger cardinalities have the consequence, that each
  category affects less nodes. For that reason, one can be less vigilant when
  assigning link-affinity probabilities for these attributes. Having discussed
  that, gender did generated some mild clustering within the network. This 
  contributed to the network structure and was successfully exploited using 
  graph machine learning. If gender was not considered for generating the graph, 
  the accuracy of the GraphSage model was slightly decreased. \\

  \noindent The graphs shown in figure \ref{fig:us_airline_nodes} thus provide 
  very useful insights for better understanding the relationships within 
  the data. This is a welcome bi-product of the \acs{mag} model and it is surprising 
  to see how informative the resulting graph is for analyzing the data. In fact, 
  it provides vastly superior insights compared to what a standard scatter plot 
  can reveal given the available feature data. For reference, the pairplot of 
  the feature data can be found in appendix \ref{App:pairplot}. In this regard, 
  the \acs{mag} model is very exciting for visually analyzing data. \\

  \noindent Applying methods such as Node2Vec further allow for the creation
  of 2-dimensional node embeddings which can be used to create scatter plots. The
  scatter plots illustrated in figure \ref{fig:node2vec} show the node embeddings 
  of the graph shown in figure \ref{fig:us_airline_nodes}. The scatter plot further
  helps to solidify the insights gained from the graph. A limitation of the
  graph for visual interpretation is, that one technically would need to
  consider the node connections. The large size of the graph however makes it
  impossible to visually follow the node connections for interpretation. Node2Vec
  considers the node connections/edges of the network and removes the mentioned
  limitation for visually interpreting the graph. Figure \ref{fig:node2vec}
  shows, that the visual insights gained from the graph are still valid after
  considering the node connections. The Node2Vec algorithm is a beautiful
  dimensionality reduction technique, which successfully reduced the complexity
  of the graph shown in figures \ref{fig:us_airline_graph} \& 
  \ref{fig:us_airline_nodes}. In this sense, the \acs{mag} model in combination 
  with \acs{grl} such as Node2Vec can be used as powerful cluster analysis tools.
  
  \section{Appropriateness of Semi-Synthetic Graphs for Machine Learning}
  \label{section:gml_performance}

  The discussion in the previous section shows, that semi-synthetic graphs
  themselves can provide valuable visual insights. In this section it is
  discussed to what extent semi-synthetic graphs are appropriate for \acs{gml}. 
  The perhaps most instructive evidence is provided by the bank telemarketing 
  dataset presented in section \ref{section:bank_data}. \acs{gml} fails to 
  perform well on the graph generated from this dataset. It is shown, that the 
  attributes chosen for the \acs{mag} model are unsuccessful for generating 
  network structures that provided useful additional information which then 
  could be exploited using \acs{gml}. The bank telemarketing dataset is 
  notoriously difficult due to the unbalanced distribution of the label. The 
  dataset consists of approximately 88\% of observations which did not invest 
  in the short-term deposit compared to only 12\% of the observations which did 
  invest. For \acs{gml} and also the standard \acs{ml} models, it is loss 
  optimizing to classify most observations as non-investors. When generating 
  the biased graph shown in figure \ref{fig:Moro_bias}, a classification 
  accuracy of > 95\% is achieved which is a remarkable improvement. As 
  mentioned in section \ref{section:bank_data}, the biased graph which includes 
  the label as an attribute cannot be used in a practical setting and is a form of
  cheating. Nevertheless it shows, that if a network structure can be generated
  which captures the label well, the problem of unbalanced label data can be
  overcome. In particular it shows, that the attributes must be capable of
  generating network structures that are at least in part predictive of the
  label. Overcoming the problem presented by unbalanced label data is a relevant 
  issue for \acs{ml} and warrant further research. \\

  \noindent The \acs{us} airline passenger dataset is shown to be well suited for 
  \acs{gml}. The reasons for its success are in line with the previous 
  discussion regarding the bank telemarketing dataset. The attributes used for 
  the \acs{mag} model generated network structures which are in part predictive 
  for the label satisfaction of the \acs{us} airline passenger dataset. This is 
  shown when looking at the satisfaction plots in figures 
  \ref{fig:us_airline_nodes} \& \ref{fig:node2vec}. \\ 

  \noindent The graph structure alone yields poor results when looking at the 
  results shown in section \ref{section:result_n2v} using Node2Vec. Taking a 
  step back with a more generous perspective, it is nevertheless impressive that 
  an accuracy of approximately 76\% could be achieved for the training- and 
  validation dataset which only considers 2-dimensional node embeddings. This 
  however did not translate to the test graph for which the best accuracy was at 
  only approximately 58\%. In this sense, the model is only marginally more 
  successful at the classification task than randomly guessing the 
  classification. \\

  \noindent \acsp{gcn} also achieve a classification accuracy of approximately 
  76\%. This result is even worse when comparing it to Node2Vec. The \acs{gcn} 
  considers the network structure as well as the node features and should 
  therefore be capable of outperforming Node2Vec. The \acs{gcn} model clearly 
  fails here and has the serious limitation that it cannot be applied in an 
  inductive setting to unseen graphs. \acs{gcn} is a breakthrough model 
  as it is one of the first successful \acsp{gnn}. Nevertheless, it 
  performed poorly for the given datasets and is surpassed in terms of 
  performance by more modern \acs{gnn} models such as GraphSage. \\

  \noindent The Final \acs{gml} model tested is GraphSage. This model yields 
  very good results as shown in section \ref{section:graphsage_results}. 
  GraphSage is capable of training models which have a test accuracy of > 94\%. 
  The aggregation strategies used for the GraphSage model are successful in the 
  order max-pooling $\geqslant$ sum-pooling > LSTM > mean. This ordering is 
  based on the results presented in section \ref{section:graphsage_results} and 
  is confirmed by the simulation results shown in section 
  \ref{section:graphsage_simulation} and appendix \ref{App:sim_results}. 
  When comparing the two main aggregation strategies of interest, max-pooling 
  and sum-pooling, both methods almost yield identical test results. For the
  results shown in section \ref{section:graphsage_results} max-pooling is only 
  more successful in classifying 1 more passenger as satisfied compared to
  sum-pooling. The test results are otherwise identical as shown in tables 
  \ref{table:max_results_test} \& \ref{table:sum_results_test}. The simulation
  results with 100 experiments for both max-pooling and sum-pooling confirm,
  that max-pooling yields marginally better results for the test graph with an
  average accuracy of 93.80\% compared to 93.76\% for sum-pooling. Sum-pooling
  is however more consistent, whereas max-pooling has a larger variance 
  regarding the test accuracy. In addition, max-pooling suffered from two large 
  negative test accuracy outliers during the simulation. Sum-pooling also 
  suffered from two negative test accuracy outliers, which were however less 
  pronounced. The GraphSage results show, that semi-synthetic graphs can indeed 
  be appropriate for \acs{gml}.

  \section[GML vs. Standard ML]{Graph Machine Learning vs. Standard Machine Learning}
  \label{section:comp_disc}

  The overview table \ref{table:result_comparison} shows that the results using 
  GraphSage are competitive in comparison to the standard \acs{ml} models. The 
  results further confirm that the results using Node2Vec and the GCN are not 
  competitive. GraphSage is proven to be the second best method behind the 
  random forest classifier. The \acs{ann} model on average tends to yield 
  marginally inferior test accuracy results as presented in the simulation results 
  shown in appendix \ref{App:ann_sim}. In addition, the \acs{ann} is more prone to 
  over-fitting as illustrated in figure \ref{fig:ANN_fit}. With regard to the other 
  standard \acs{ml} methods, GraphSage clearly outperformed these methods. \\

  \noindent This shows, that semi-synthetic graphs can be used for \acs{gml} in 
  a competitive setting. The accuracy improvements one would hope for 
  considering the rather tedious and difficult graph generation procedure is 
  however not achieved. As shown in table \ref{table:result_comparison}, a 
  simpler models such as the random forest classifier consistently outperforms 
  the GraphSage model. In addition, a simpler standard \acs{ann} yields very 
  similar results as GraphSage. In \acs{ml} the principle of Occam's razor is 
  often referred to for choosing appropriate models. Following this principle, 
  simpler models are preferred if they yield similar/identical results compared 
  to more complex methods.GraphSage using semi-synthetic graphs is most 
  definitely the most complex method shown in table \ref{table:result_comparison}. 
  For that reason, certainly the random forest classifier is preferred as it 
  yields better accuracies and is a more simple model compared to GraphSage. In 
  addition, one could argue that the \acs{ann}, \acs{svm}, and AdaBoost are 
  preferential to the GraphSage model as well. These models are simpler with 
  only marginally inferior results. For the evaluation one needs to also consider 
  the context of the classification task. In a medical context, accuracies are 
  of utmost importance where one is willing to use more complex methods to 
  receive marginal accuracy improvements. For the purpose of gaining customer 
  insights, the consequences for having marginally higher- or lower accuracies 
  are less severe. In such a setting, the principle of Occam's razor can be more 
  fully embraced which is why the \acs{ann}, \acs{svm}, and AdaBoost might be 
  preferred as they are simpler and more practical for application in a 
  real world setting. This is a subjective assessment and depending on the 
  context, it might differ. Nevertheless, GraphSage yields the second best 
  accuracies and remains a competitive model. \\ 

  \noindent Last but not least, the comparison of real network properties and
  the \acs{mag} presented in section \ref{section:airline_data} must be
  discussed. As shown, the graph generated using the \acs{us} airline passenger
  dataset is not in line with the properties of real networks. At first glance,
  this might not appear to be of importance. The results presented in chapter
  \ref{section:results} show, that \acs{gml} was competitive but not superior
  to standard \acs{ml} models. The network structure appears to only provide a
  marginal improvement for the \acs{gml} models. A reason for this limited
  success could be, that the vertex degrees are too large. This for instance has
  the consequence, that the GraphSage model has the same network structure at
  every layer. According to the model specifications, the GraphSage model
  samples 10 nodes at the 2-hop distance and 5 nodes at the 1-hop distance.
  Given, that the degree distribution shown in figure
  \ref{fig:centrality_measures} starts at approximately 150, the GraphSage will
  always sample exactly 10 and 5 nodes respectively. In this setting, only the 
  feature data and the general neighborhood of the nodes can be considered.
  Specific network structures withing the neighborhoods can however not be
  exploited. This points to a serious limitation of the generated \acs{mag} 
  and might be a reason why GraphSage did not perform better. A similar
  observation is also made for the \acs{gcn}. This model always has to consider
  the entire graph Laplacian. Given the large starting value of the degree
  distribution, this can become a challenging task. While, the model may be
  able to differentiate general neighborhoods, recognizing more local
  neighborhoods could prove to be excessively challenging. This is an area which
  should be investigated for future research.






