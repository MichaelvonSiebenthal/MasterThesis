
  The analysis of the data in chapter \ref{section:data} and the presentation
  of the machine learning results in chapter \ref{section:results} provided a
  myriad of interesting results. These results primarily touch following topics
  and provide the structure for the discussion:

  \begin{enumerate}
      \item Informativeness of semi-synthetic graphs
      \item Appropriateness of semi-synthetic graphs for machine learning
      \item Graph machine learning versus standard machine learning 
  \end{enumerate}

  \section{Informativeness of Semi-Synthetic Graphs}
  
  The graph generated for the US Airline Passenger data set presented in section
  \ref{section:airline_data} provided interesting results in term of cluster
  analysis. The MAG model was successfully used for generating a graph that
  captured useful neighborhood structures as shown in figure
  \ref{fig:us_airline_nodes}. Specifically, similar nodes were grouped together 
  based on their similarity given the selected attributes for the MAG model. 
  Especially appealing is that neighborhoods are also consistent within 
  neighborhoods. To illustrate this, one can observe that all nodes are
  clustered together based on their similarity even within the two main
  clusters which formed based on the attribute type of travel. This
  observation is consistent for several additional levels. For instance, nodes
  form neighborhoods based on age within the cluster of business class
  passengers which is part of the cluster of passenger which are traveling for
  business purposes. The only attribute for which no significant clusters
  emerged is the attribute gender. The reason for this is, that the
  link-affinity matrix formed a connection with probability of 0.6 for
  observations with the same gender and 0.4 if they were of the opposite
  gender. These probabilities were set based on the assumption that
  observations of the same gender are more similar than the opposite gender. It
  is however not assumed, that the passenger experience for women would be
  significantly different to the one for men. In addition, one needs to be
  careful when assigning link-affinity probabilities. The data is almost
  perfectly 50/50 split into male and female observations. If one assigns very
  low- or high probabilities, this will significantly alter the graph
  generation of all nodes. This was for instance done for the attribute type of 
  travel which split the graph into two main clusters. This is less of a concern 
  for variables with multiple categories such as age. Nevertheless, the
  attribute gender did generate some mild clustering within the network, which 
  could be exploited for graph machine learning. One can observe this mild
  clustering in the graph, the effect is however not pronounced enough to draw
  any conclusions from visual observations. The graphs shown in figure
  \ref{fig:us_airline_nodes} thus provide very useful insights for better
  understanding the relationships within the data. This is a welcome bi-product
  of the MAG model and it is surprising to see how informative the resulting 
  graph is for analyzing the data. In fact, it provided vastly superior 
  insights than what a standard scatter plot could reveal given the available 
  feature data. For reference, the pairplot of the feature data can be found in 
  appendix \ref{App:pairplot}. In this regard, the MAG model is very exciting for visually
  analyzing data. \\

  \noindent Applying methods such as Node2Vec further allow to create
  2-dimensional node embeddings which can be used to create scatter plots. The
  scatter plots shown figure \ref{fig:node2vec} show the node embeddings of the
  graph shown in figure \ref{fig:us_airline_nodes}. The scatter plot further
  helps to solidify the insights gained from the graph. A limitation of the
  graph for visual interpretation is, that one technically would need to
  consider the node connections. The large size of the graph however makes it
  impossible to follow the node connections for interpretation. Node2Vec
  considers the node connections/edges of the network and removes the mentioned
  limitation of visually interpreting the graph. Figure \ref{fig:node2vec}
  shows, that the visual insights gained from the graph are still valid after
  considering node connections. The Node2Vec algorithm is a beautiful
  dimensionality reduction technique, which successfully reduced the complexity
  of the graph shown in figures \ref{fig:us_airline_graph} \& 
  \ref{fig:us_airline_nodes}. In this sense, the MAG model in combination with
  Graph Representation Learning such as Node2Vec can be used as powerful
  clustering tools.
  
  \section{Appropriateness of Semi-Synthetic Graphs for Machine Learning}
  \label{section:gml_performance}

  The discussion in the previous section revealed, that semi-synthetic graphs
  themselves can provide valuable visual insights. In this section it is
  discussed to what extent semi-synthetic graphs are appropriate for graph
  machine learning. The perhaps most instructive evidence is provided by the Bank
  Telemarketing data set presented in section \ref{section:bank_data}. Graph
  machine learning failed to perform well on the graph generated form this
  dataset. It was shown, that the attributes chosen for the MAG were
  unsuccessful for generating network structures that provided useful additional
  information which could be exploited using graph machine learning. The Bank
  Telemarketing data set is notoriously difficult due to the unbalanced
  distribution of the label. The dataset consist of approximately 90\% of
  observations which did not invest in the long-term deposit compared to only 
  10\% of the observations which did invest. For graph machine learning and
  also the standard machine learning methods, it was loss optimizing to
  classify most if not all observations as non-investors. When generating the
  biased graph shown in figure \ref{fig:Moro_bias}, a classification accuracy
  of > 95\% was achieved which is remarkable improvement compared to the
  accuracies using the unbiased graph shown in figure \ref{fig:Moro} and the
  standard machine learning methods which ranged between 87-89\%. As outlined
  in section \ref{section:bank_data}, the biased graph which includes the label
  as an attribute cannot be used in a practical setting and is a form of
  cheating. Nevertheless it shows, that if a network structure can be generated
  which captures the label well, the problem of unbalanced label data can be
  overcome. In particular it shows, that the attributes must be capable of
  generating network structures that are at least in part predictive of the
  label. This is an interesting topic and warrants further research, as
  overcoming the problem presented by unbalanced label data is a relevant issue
  for machine learning. \\

  \noindent The US Airline Passenger data set yielded very good results in
  terms of graph generation and subsequent use for graph machine learning. The 
  reasons for its success are in line with the previous discussion regarding
  the Bank Telemarketing data set. The attributes used for the MAG model
  generated network structures which were in part predictive for the label
  satisfaction of the US Airline Passenger data set. This is shown when looking
  at the satisfaction plots in figures \ref{fig:us_airline_nodes} \&
  \ref{fig:node2vec}. \\ 

  \noindent The graph structure alone yielded poor results when looking at the 
  results shown in section \ref{section:result_n2v} using Graph Representation 
  learning. Taking a step back with a more generous perspective, it is 
  nevertheless impressive that an accuracy of approximately 76\% could be 
  achieved for the training- and validation dataset which only considers 
  2-dimensional node embeddings. This however did not translate to the test 
  graph for which the best accuracy was at only approximately 58\%. In this 
  sense, the model was only marginally more successful at the classification 
  task than randomly guessing the classification. \\

  \noindent Graph Convolutional Networks also achieved a classification 
  accuracy of approximately 76\%. This result is even worse when comparing it
  to Graph Representation Learning using Node2Vec. The GCN considers the
  network structure as well as the node features and should therefore be
  capable of outperforming Node2Vec. The GCN model clearly fails here and has
  the serious limitation that it cannot be applied in an inductive setting to
  unseen graphs. GCN is a breakthrough model as it is one of the first
  successful Graph Neural Networks. Nevertheless, it performed poorly for the
  given datasets and is surpassed in terms of performance by more modern GNN
  methods such as GraphSage. \\

  \noindent The Final graph machine learning method applied was GraphSage. This
  method yielded very good results as shown in section
  \ref{section:graphsage_results}. GraphSage is capable of training models
  which have a test accuracy of > 94\%. The aggregation strategies used for the
  GraphSage model further are successful in the order max-pooling $\geqslant$ 
  sum-pooling > LSTM > mean. This ordering is based on the results presented in 
  section \ref{section:graphsage_results} and is confirmed by the simulation 
  results shown in section \ref{section:graphsage_simulation} and appendix
  \ref{App:sim_results}. 
  When comparing the two main aggregation strategies of interest, max-pooling 
  and sum-pooling, bot methods almost yielded identical test results. For the
  results shown in section \ref{section:graphsage_results} max-pooling was only 
  more successful in classifying 1 more passenger as satisfied compared to
  sum-pooling. The test results are otherwise identical as shown in tables 
  \ref{table:max_results_test} \& \ref{table:sum_results_test}. The simulation
  results with 100 experiments for both max-pooling and sum-pooling confirm,
  that max-pooling yields marginally better results for the test graph with an
  average accuracy of 93.80\% compared to 93.76\% for sum-pooling. Sum-pooling
  is however more consistent, whereas max-pooling has a larger variance
  regarding the variance of the test accuracy. In addition, max-pooling
  suffered from two large negative outliers in terms test accuracy. Sum-pooling
  also suffered from two negative outliers, which were however less pronounced.
  The GraphSage results show, that semi-synthetic graphs can indeed be
  appropriate for graph machine learning.

  \section{Graph Machine Learning Versus Standard Machine Learning}

  The overview shown in table \ref{table:result_comparison} shows that the
  results using GraphSage are competitive in comparison to the standard machine
  learning methods. The results further confirm that the results using Graph
  Representation Learning and the GCN are not competitive. GraphSage is shown
  to be the second best method behind the Random Forest classifier. The ANN
  model on average tends to yield marginally inferior test accuracy results as
  shown in the simulation results shown in appendix \ref{App:ann_sim}. In
  addition, the ANN is more prone to over-fitting as shown in figure
  \ref{fig:ANN_fit}. With regard to the other standard machine
  learning methods, GraphSage clearly outperformed these methods. \\

  \noindent As shown, the GraphSage model is a competitive method for 
  classifying the US Airline Passenger dataset. This shows, that semi-synthetic
  graphs can be used for graph machine learning in a competitive setting. The
  accuracy improvements one would hope for considering the rather tedious and
  difficult graph generation procedure was however not achieved. As shown in
  table \ref{table:result_comparison}, a simpler method such as the Random
  Forest Classifier consistently outperformed the GraphSage model. In addition
  a simple standard ANN yields very similar results as GraphSage. In machine 
  learning the principle of Occam's razor is often referred to for choosing 
  appropriate methods. Following this principle, simpler models are preferred if 
  they yield similar/identical results compared to more complex methods.
  GraphSage using semi-synthetic graphs is most definitely the most complex 
  method shown in table \ref{table:result_comparison}. For that reason certainly 
  the Random Forest Classifier is preferred as it yields better accuracies and
  is a more simple model compared to GraphSage. In addition, one could argue
  that the ANN, SVM and AdaBoost are preferential to the GraphSage model as
  well, as they are simpler models with only marginally inferior results. For
  the evaluation one needs to also consider the context of the classification
  task. In a medical context, accuracies are of utmost importance where one is
  willing to use more complex methods to receive marginal accuracy improvements. 
  For the purpose of gaining customer insights, the consequences for having
  marginally higher- or lower accuracies are less severe. In such a setting,
  the principle of Occam's razor can be more fully embraced which is why the
  ANN, SVM and AdaBoost might be preferred as they are simpler and more
  practical for application in a real-world setting. Nevertheless, GraphSage
  yielded the second best accuracies and could be the preferred method for a
  different setting.  





