
  \onehalfspacing

  This master's thesis investigates graph machine learning for the purpose of
  gaining customer insights. Given that customer data is most commonly only
  available as standard cross-sectional data, semi-synthetic graph generation is
  investigated. Specifically it is assessed to what extent semi-synthetic graphs
  can be utilized for graph machine learning. To assess the viability and
  competitiveness of this approach, different graph machine learning methods
  are tested. In addition, the graph machine learning results are compared to
  the results using common standard machine learning methods. The results of
  the GraphSage method reveal, that using semi-synthetic graphs can indeed be a
  viable approach. The semi-synthetic graphs can further provide useful results
  for visualizing relationships within the data. Lastly it is shown, that the
  semi-synthetic graph could potentially be used for overcomming the
  difficulties associated with unbalanced label data. It is shwon, that 
  semi-synthetic graphs are limited to the extent that they can generate useful 
  graph structures for predicting the label of a given classification task. The
  uncommon graph properties of semi-synthetic graphs further limit the
  capabilites of some graph machine learning methods. Suggestions for future 
  research are provided which focus on improving semi-syntetic graph generation, 
  using graphs for cluster analysis and improving graph machine learning methods.
