
  \onehalfspacing

  This master's thesis investigates graph machine learning for the purpose of
  gaining customer insights. Given that customer data is most commonly only
  available as standard cross-sectional data, semi-synthetic graph generation is
  investigated. Specifically it is assessed to what extent semi-synthetic graphs
  can be utilized for graph machine learning. To assess the viability and
  competitiveness of this approach, different graph machine learning models
  are tested. In addition, the graph machine learning results are compared to
  the results of common standard machine learning models. The results of
  the GraphSage model reveals, that using semi-synthetic graphs can indeed be a
  viable approach. The semi-synthetic graphs can further provide useful results
  for visualizing relationships within the data. Lastly it is shown, that the
  semi-synthetic graph could potentially be used for overcoming the
  difficulties associated with unbalanced label data. Semi-synthetic graphs are 
  are shown to be limited by the extent to which they can capture useful graph 
  structures for predicting the label of a given classification task. The
  uncommon graph properties of semi-synthetic graphs further limit the
  capabilities of some graph machine learning models. Suggestions for future 
  research is provided and focuses on improving semi-synthetic graph generation, 
  using graphs for cluster analysis and improving graph machine learning models.
